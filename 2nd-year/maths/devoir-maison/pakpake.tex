%-------------------------------------------------------------------------------
% Title:   DM2 - Math
% Authors: pakpake
% Created: 5 avril 2020
% Last Revision: 7 avril 2020
%-------------------------------------------------------------------------------
\documentclass[a4paper,12pt]{article}

\usepackage[french]{babel}
\usepackage[utf8]{inputenc}
\usepackage[T1]{fontenc}
%\usepackage{lmodern}

\usepackage{textpos}
\usepackage[a4paper,left=2.5cm, right=2.5cm, top=2.5cm, bottom=3.5cm]{geometry}
\usepackage[]{amsmath,amssymb}

\setlength{\parindent}{0ex}
\setlength{\parskip}{1ex}

\renewcommand{\thesection}{Exercice \arabic{section}}
\renewcommand{\thesubsection}{Partie \Roman{subsection}.}
\renewcommand{\thesubsubsection}{\arabic{subsubsection}.}

\DeclareMathOperator{\Ker}{Ker}
\DeclareMathOperator{\Img}{Im}
\DeclareMathOperator{\vect}{vect}
\DeclareMathOperator{\Mat}{\mathcal{M}at}
%\DeclareMathOperator{\det}{det}
\DeclareMathOperator{\M}{\mathcal{M}}
\DeclareMathOperator{\B}{\mathcal{B}}
\DeclareMathOperator{\E}{\mathcal{E}}
\DeclareMathOperator{\di}{dim}

 
\title{Devoir Maison de mathématiques n$^\circ$ 2}
\author{pakpake}
\date{7 Avril 2020}


\begin{document}

\maketitle

% Exercice 1
\section{}

% Partie I
\subsection{}

% Question 1
\subsubsection{Montrons que $f$ est linéaire :}

\begin{align*}
f(P) &= (X+1)P
\end{align*}

$\forall P, Q \in \mathbb{R}^2[X]$, on a :

\begin{align*}
  f(P+\lambda Q)&=(X+1)(P+\lambda Q) \\
                &= (X+1)P + (X+1)\lambda Q \\
                &= (X+1)P + \lambda (X+1)Q \\
                &= f(P) + \lambda f(Q)
\end{align*}

Donc $f$ est une application linéaire.

% Question 2
\subsubsection{Déterminons le noyau et l'image de $f$}

On peut écrire $P(X)$ sous la forme $P(X)=aX^2+bX+c$

On a donc :
\begin{align*}
  f(P)         & = (X+1)P                 \\
  f(aX^2+bX+c) & = aX^3+bX^2+cX+aX^2+bX+c \\
               & = aX^3+(b+a)X^2+(c+b)X+c
\end{align*}

On peut maintenant résoudre $f(P)=0$

\[
  \left\{ \begin{array}{rl}
    a   & = 0 \\
    b+a & = 0  \\
    c+b & = 0  \\
    c   & = 0
  \end{array}\right.
  \Leftrightarrow
  \left\{ \begin{array}{rl}
    a & =0 \\
    b & =0 \\
    c & =0
  \end{array} \right.
\]

Ce qui implique que $\Ker f = \vect(0)$

Ainsi $f$ est injective.

On a d'après le théorème du rang :
\[
  \di E_3 = \di \Ker f + \di \Img f
\]

Avec $E_3$ l'espace des polynômes

Or $\di \Ker f = 0$ et $\di E_3 = 4$, donc $\di \Img f= 4$

On a ainsi $\Img f = \vect{(1,X,X^2,X^3)}$

Donc $\di \Img f = \di\mathbb{R}^3 = 4$

Ainsi, $f$ est surjective.

Pour conclure, $f$ est injective et surjective, $f$ est donc bijective.

% Question 3
\subsubsection{}

\[
  \begin{array}{lll}
    f(1)   & =(X+1)1   & =X+1     \\
    f(X)   & =(X+1)X   & =X^2+X   \\
    f(X^2) & =(X+1)X^2 & =X^3+X^2
  \end{array}
\]

% Question 4
\subsubsection{}

\[
  \begin{array}{ccc}
      &  \begin{array}{ccc} f(1)&f(X)&f(X^2) \end{array} & \\
  A = & \left(\begin{array}{c@{\hspace*{3em}}c@{\hspace*{3em}}c}
            1 & 0 & 0 \\
            1 & 1 & 0 \\
            0 & 1 & 1 \\
            0 & 0 & 1
          \end{array} \right) 
          & \begin{array}{c} e_1\\ e_2 \\ e_3 \\ e_4 \end{array}
  \end{array}
  = \mathcal{M}_{\mathbb{R}^2[X],\mathbb{R}^3[X]}(A)
\]

% Partie II
\subsection{}

% Question 1
\subsubsection{Montrons que $g$ est linéaire :}

D'après la définition de dérivation, on a :

\[ 
  (P+Q)'(1) = P'(1)+Q'(1)
\]

et,

\[ 
  (\lambda P)'(1)=\lambda P'(1)
\]

Ainsi, on a : % IL FAUT EN RAJOUTER UN PEU ???

\[
  g(P+\lambda Q) = g(P)+g(\lambda Q) = g(P) + \lambda g(Q)
\]

Donc $g$ est linéaire.

% Question 2
\subsubsection{}

Sur $\mathbb{R}^3[X]$ :

\begin{align*}
  P(X)   & = aX^3 +bX^2+cX+d \\
  P'(X)  & = 3aX^2+2bX+c     \\
  P''(X) & = 6aX+2b         
\end{align*}

On a donc

\begin{align*}
  P(1)   & = a+b+c+d \\
  P'(1)  & = 3a+2b+c  \\
  P''(1) & = 6a+2b
\end{align*}

Ainsi $\Ker g \Leftrightarrow \{(P(1);P'(1);P''(1))\} = {(0,0,0)}$

\[
  \left\{\begin{array}{rl}
    a+b+c+d & = 0 \\
    3a+2b   & = 0 \\
    6a+2b   & =0
  \end{array} \right.
  %
  \Leftrightarrow
  %
  \left\{\begin{array}{rl}
    a+b+c+d & = 0                             \\
    2b      & = -3a-c                         \\
    a       & = \frac{-2b}{6} = \frac{-1}{3}b
  \end{array}\right.
  %
  \Leftrightarrow
  %
  \left\{\begin{array}{rl}
    a+b+c+d & = 0             \\
    2b      & = b-c           \\
    a       & = \frac{-1}{3}b
  \end{array}\right.
\]

\[
  \Leftrightarrow
  %
  \left\{\begin{array}{rl}
    c & = \frac{1}{3}b +c-d \\
    b & = -c                \\
    a & = \frac{-1}{3}b
  \end{array}\right.
  %
  \Leftrightarrow
  %
  \left\{\begin{array}{rl}
    a & = \frac{1}{3}c  \\
    b & = -c            \\
    d & = \frac{-1}{3}c
  \end{array}\right.
\]

On a donc $\Ker g = \vect{(\frac{1}{3};-1;1;-\frac{1}{3})} \\$

On a $\di \Ker g = 1$ et de plus, $\Ker g$ n'est pas égal au vecteur nul, donc
on en déduit que $g$ n'est pas injective.

De plus, $\di E \neq \di \Img g$

avec $\di E = 4$ et $\di \Img g = 3$

On peut donc écrire $\Img g = \{(X^2;X;1)\}$

Ainsi on peut dire que $g$ n'est pas surjective. Donc, $g$ n'est pas bijective.


% Question 3
\subsubsection{}

On peut calculer $g(1),g(X),g(X^2) \mbox{ et } g(X^3)$

%J'AI CORRIGE On a $\Img g = \vect{(1,,X^2,X^3)}$
On a $\Img g = \vect{(1,X,X^2,X^3)}$

On a donc : 

\begin{align*}
  g(1)   & =(1,0,0)       \\
  g(X)   & =(X,1,0)       \\
  g(X^2) & =(X^2,2X,2)    \\
  g(X^3) & =(X^3,3X^2,6X)
\end{align*}
$\Leftrightarrow$
\begin{align*}
    g(1)&=1 \\
    g(X)&=1+X\\
    g(X^2)&=2+2X+X^2\\
    g(X^3)&=6X+3X^2+X^3
\end{align*}

On a ainsi la matrice suivante :


\[
  \begin{array}{cccc}
    &\begin{array}{cccc} g(1)&g(X)&g(X^2)&g(X^3) \end{array} & \\
    B=&\left(\begin{array}{c@{\hspace*{3em}}c@{\hspace*{3em}}c@{\hspace*{3em}}c}
            1 & 1 & 2 & 0 \\
            0 & 1 & 2 & 6 \\
            0 & 0 & 1 & 3 \\
            0 & 0 & 0 & 1
          \end{array} \right)
  \end{array}
  = \Mat g
\]

% Partie III
\subsection{}

On a :

\begin{align*}
  h        & =g\circ f                                             \\
  g\circ f & =\Mat g \times \Mat f = \Mat C
\end{align*}

\[
  \Leftrightarrow
  %
  \begin{pmatrix}
    1 & 1 & 2 & 0 \\
    0 & 1 & 2 & 6 \\
    0 & 0 & 1 & 3 \\
    0 & 0 & 0 & 1
  \end{pmatrix}
  .
  \begin{pmatrix}
    1 & 0 & 0 \\
    1 & 1 & 0 \\
    0 & 1 & 1 \\
    0 & 0 & 1
  \end{pmatrix}
  =
  \begin{pmatrix}
    2 & 3 & 2 \\
    1 & 3 & 8 \\
    0 & 1 & 4 \\
    0 & 0 & 1
  \end{pmatrix}
\]

\clearpage

% Exercice 2
\section{}

% Question 1
\subsubsection{}

On a :

\[
  \mathcal{M}_{\varepsilon}(f)=
  \begin{array}{ccc}
    &\left(\begin{array}{c@{\hspace*{3em}}c@{\hspace*{3em}}c}
            2 & 0 & -1 \\
            0 & 3 & 0  \\
            1 & 2 & 1
          \end{array} \right) 
    & \begin{array}{c} e_1\\ e_2 \\ e_3 \end{array}
  \end{array}
\]

% Question 2
\subsubsection{}

Calcul du déterminant :

$\det \mathcal{M}(f) = 3(2-(1\times(-1))) = 3(2+1) = 9 \neq 0$

Comme le déterminant est non-nul, on peut dire que la matrice est inversible, et que les vecteurs colonnes forment une base.

Aussi, comme $\det\ \mathcal{M}(f) \neq 0$, on peut dire que l'application $g$ est injective.

De plus, on sait que le rang d'une matrice $n\times n$ constituée des 3 vecteurs de la base, est $n$. % 

D'où ici, $rg(f)=3$, ainsi, $f$ est surjective.

Et comme, $g$ est injective et surjective, $g$ est bijective.

% Question 3
\subsubsection{}

Soient les vecteurs : \[ b_1=(1,0,1),\quad b_2=(1,1,1),\quad b_3=(-1,1,0).\]

% Question 4
\subsubsection{Montrons que la famille $\mathcal{B}=\{b_1,b_2,b_3\}$ est une base.}

On obtient la matrice suivante :


$$
\begin{array}{ccc}
  &\begin{array}{ccc} b_1&b_2&b_3 \end{array} & \\
  \mathcal{M}_{\mathcal{B}}=&\left(\begin{array}{c@{\hspace*{1em}}c@{\hspace*{1em}}c}
            1&1&-1\\
            0&1&1\\
            1&1&0 \end{array} \right) 
\end{array}
$$

On peut ainsi calculer le déterminant :

\begin{align*}
det\ \mathcal{B}&=1(1\times0-1\times1) \\
&=1(0-1) \\
&=(-1)\neq 0
\end{align*}

Donc, la matrice est inversible, et on peut dire que la famille $\mathcal{B}=\{b_1,b_2,b_3\}$ forme une base.

% Question 5
\subsubsection{}

La matrice de passage $\mathcal{M}_{\mathcal{B,E}}(id)$ est tout simplement :

$$
\begin{array}{ccc}
  P=\mathcal{M}_{\mathcal{B,E}}(id)=&\left(\begin{array}{c@{\hspace*{1em}}c@{\hspace*{1em}}c}
            1&1&-1\\
            0&1&1\\
            1&1&0 \end{array} \right) 
\end{array}
$$

Pour obtenir $\mathcal{M}_{\mathcal{E,B}}(id)$, il suffit de faire $P^{-1}.$

Or, $P^{-1}=\frac{1}{\mbox{det}\ P}\times ^t\mbox{Com}\ P$

Ce qui nous donne :

$$
\begin{array}{ccc}
    P^{-1}=\mathcal{M}_{\mathcal{E,B}}(id)=&\left(\begin{array}{c@{\hspace*{1em}}c@{\hspace*{1em}}c}
            -1&-1&2\\
            1&1&-1\\
            -1&0&1 \end{array} \right) 
\end{array}
$$

% Question 6
\subsubsection{Déterminons la matrice  $\M_{\B,\B}(f)$}

On a,

\begin{align*}
    f(x_1)&=\alpha\vec{b_1}+\beta\vec{b_2}+\gamma\vec{b_3}=(2,0,1) \\
    &=\alpha(1,0,1)+\beta(1,1,1)+\gamma(-1,1,0) \\
    &=(\alpha+\beta-\gamma,\beta+\gamma,\alpha+\beta)
\end{align*}

On peut donc procéder par identification via un système :

$$
    \left\{\begin{array}{rl}
        \alpha+\beta-\gamma &=2 \\
        \beta+\gamma &= 0 \\
        \alpha+\beta&=1
    \end{array} \right.
    %
    \Leftrightarrow
    %
    \left\{\begin{array}{rl}
        \alpha-\gamma-\gamma &=2 \\
        \beta &= -\gamma \\
        \alpha-\gamma &= 1 \Leftrightarrow \gamma=-1+\alpha
    \end{array} \right.
    %
    \Leftrightarrow
    %
    \left\{\begin{array}{rl}
        \alpha-2(-1+\alpha)&=2 \\
        \alpha+2-2\alpha&=2 \\
        -\alpha&=0
    \end{array}\right.
    %
$$
$$
    \Leftrightarrow
    %
    \left\{\begin{array}{rl}
        \alpha&=0 \\
        \beta&=1 \\
        \gamma&=-1
    \end{array}\right.
    \Leftrightarrow
    %
    f(x_1)=\begin{pmatrix}0\\1\\-1\end{pmatrix}
$$

Donc $f(x_1)=\vec{b_2}-\vec{b_3}$
\clearpage

$$f(x_2)=\alpha\vec{b_1}+\beta\vec{b_2}+\gamma\vec{b_3}=(0,3,2)$$

$$
    %
    \left\{\begin{array}{rl}
        \alpha+\beta-\gamma&=0 \\
        \beta+\gamma&=3 \\
        \alpha+\beta&=2
    \end{array}\right.
    %
    \Leftrightarrow
    %
    \left\{\begin{array}{rl}
        \alpha+3-\gamma-\gamma&=0 \\
        \beta&=3-\gamma \\
        \alpha+3-\gamma&=2 
    \end{array}\right.
$$
$$
    %
    \Leftrightarrow
    %
    \left\{\begin{array}{rl}
        -1+\gamma+3-\gamma-\gamma&=0 \\
        \beta&=3-\gamma \\
        \alpha&=-1+\gamma
    \end{array}\right.
    %
    \Leftrightarrow
    %
    \left\{\begin{array}{rl}
        \alpha&=1 \\
        \beta&=1 \\
        \gamma&=2
    \end{array}\right.
    %
    \Leftrightarrow
    %
    f(x_2)=\begin{pmatrix}1\\1\\2\end{pmatrix}
$$

Donc $$f(x_2)=\vec{b_1}+\vec{b_2}+2\vec{b_3}$$

$$f(x_3)=\alpha\vec{b_1}+\beta\vec{b_2}+\gamma\vec{b_3}=(-1,0,1)$$

$$
    \left\{\begin{array}{rl}
        \alpha+\beta-\gamma&=-1 \\
        \beta+\gamma&=0 \\
        \alpha+\beta&=1
    \end{array}\right.
    %
    \Leftrightarrow
    %
    \left\{\begin{array}{rl}
        \alpha+\beta-\gamma&=-1 \\
        \beta&=-\gamma \\
        \alpha-\gamma&=1
    \end{array}\right.
    %
    \Leftrightarrow
$$
$$
    \left\{\begin{array}{rl}
        1+\gamma-\gamma-\gamma&=1 \\
        \beta&=-\gamma \\
        \alpha&=1+\gamma
    \end{array}\right.
    %
    \Leftrightarrow
    %
    \left\{\begin{array}{rl}
        \gamma&=2 \\
        \beta&=-2 \\
        \alpha&=3
    \end{array}\right.
$$

Donc 
\begin{align*}
    f(x_3)&=\begin{pmatrix}3\\-2\\2\end{pmatrix} \\
    &=3\vec{b_1}-2\vec{b_2}+\vec{b_3}
\end{align*}

Donc

$$
\begin{array}{ccc} 
    &\begin{array}{ccc} f(x_1)&f(x_2)&f(x_3) \end{array} & \\
    \M_{\B,\B}(f)=&\left(\begin{array}{c@{\hspace*{3em}}c@{\hspace*{3em}}c}
            0&1&3\\
            1&1&-2\\
            -1&2&2 \end{array} \right) 
\end{array}
$$
\clearpage
On a donc :
$$
%A
\begin{array}{ccc} 
\M_{\E,\E}(f)=&\left(\begin{array}{c@{\hspace*{1em}}c@{\hspace*{1em}}c}
        2&0&-1\\
        0&3&0\\
        1&2&1 \end{array} \right) 
        =A
\end{array}
$$
$$
%P^-1
\begin{array}{ccc} 
\M_{\E,\B}(id)=&\left(\begin{array}{c@{\hspace*{1em}}c@{\hspace*{1em}}c}
        -1&-1&2\\
        1&1&-1\\
        -1&0&1 \end{array} \right)
        =P^{-1}
\end{array}
$$
$$
%P
\begin{array}{ccc} 
\M_{\B,\E}(id)=&\left(\begin{array}{c@{\hspace*{1em}}c@{\hspace*{1em}}c}
        1&1&-1\\
        0&1&1\\
        1&1&0 \end{array} \right)
        =P
\end{array}
$$
\bigskip
$$
\M_{\E,\B}(f):(\mathbb{R}^3,\E)\xrightarrow[\B]{\E}(\mathbb{R}^3,\E)\xrightarrow[\M_{x\in P^{-1}}(id)]{id}(\mathbb{R}^3,\B)
$$
\vspace{4ex}

Ainsi,
\[
\M_{\E,\B}(f)=C=B\times P^{-1}
\]
\begin{align*}
    C&=\begin{pmatrix}0&1&3\\1&1&-2\\-1&2&2\end{pmatrix}\times\begin{pmatrix}-1&-1&2\\1&1&-1\\-1&0&1\end{pmatrix}\\ 
    &=\begin{pmatrix}-2&1&2\\2&0&-1\\1&3&-2\end{pmatrix}
\end{align*}

\clearpage

%Exercice 3
\section{}
% Question 1
\subsubsection{}

Soit
$$\M(f)=\begin{pmatrix}1&2&2\\0&-1&-1\\0&0&0\end{pmatrix}$$
$$det\M(f)=0$$
Donc la matrice n'est pas inversible.

$\bullet$ Déterminons le noyau de l'application linéaire f :

Pour déterminer $\Ker f$, on pose $X=\begin{pmatrix}x\\y\\z\end{pmatrix}$ et on calcule $\M(f).X=0$.

On a :

$$
\begin{pmatrix}
    1&2&2\\
    0&-1&-1\\
    0&0&0
\end{pmatrix}
.
\begin{pmatrix}x\\ y\\ z\end{pmatrix}
=\begin{pmatrix}0\\ 0\\ 0\end{pmatrix}
\Leftrightarrow
\left\{\begin{array}{rl}
    x+2y+2z&=0\\
    -y-z&=0\\
    0&=0\end{array}\right.
$$
$$
\Leftrightarrow
\left\{\begin{array}{rl}
    x&=-2y-2z\\
y&=-z\end{array}\right.
%
\Leftrightarrow
%
\left\{\begin{array}{rl}
    x&=2z-2z=0\\
y&=-z\end{array}\right.
$$

Ainsi
\begin{align*}
    \Ker f&=\{(x,y,z),x=0,y=-z,z\in\mathbb{R}\}\\
    \Ker f&=\vect{\left\{\begin{pmatrix}0\\-1\\1\end{pmatrix}\right\}}
\end{align*}

D'où $\Ker f=\vect\{-e_2+e_3\}$

$\bullet$ Déterminons l'image de l'application linéaire f :

Par le théorème du rang, on sait que :
$$\di\E=\di\Ker f+\di\Img f$$
Sachant que $\di\Ker f$ et $\di\E =3$ car ($\mathbb{R}^3$), on a donc $\di\Img f =2$.

Pour déterminer $\Img f$, on va utiliser les vecteurs de la base $\E$.

On a :
$$f\begin{pmatrix}1\\0\\0\end{pmatrix}=\begin{pmatrix}1&2&2\\0&-1&-1\\0&0&0\end{pmatrix}.\begin{pmatrix}1\\0\\0\end{pmatrix}=\begin{pmatrix}1\\0\\0\end{pmatrix}$$
Donc $\begin{pmatrix}1\\0\\0\end{pmatrix}\in\Img f$
De plus,
$$f\begin{pmatrix}0\\1\\0\end{pmatrix}=\begin{pmatrix}1&2&2\\0&-1&-1\\0&0&0\end{pmatrix}.\begin{pmatrix}0\\1\\0\end{pmatrix}=\begin{pmatrix}2\\-1\\0\end{pmatrix}$$
Donc, $\begin{pmatrix}2\\-1\\0\end{pmatrix}\in\Img f$

Or $\begin{pmatrix}1\\0\\0\end{pmatrix}$ et $\begin{pmatrix}2\\-1\\0\end{pmatrix}$ ne sont pas colinéaires, ils sont linéairement indépendants, ils engendrent donc un espace vectoriel de $\di 2$ inclus dans $\Img f$ qui est aussi de $\di 2$.

    Ainsi, on a : $\Img f=\vect\left\{\begin{pmatrix}1\\0\\0\end{pmatrix}\mbox{,}\begin{pmatrix}2\\-1\\0\end{pmatrix}\right\}$ (dans $\E$)

% Question 2
\subsubsection{}

$\bullet$ Ces espaces sont-ils supplémentaires ?

On a toujours d'après le théorème du rang $\di \E = \di \Ker f + \di \Img f$

Pour montrer que ces sous-espaces sont supplémentaires, il suffit de montrer que $\Ker f\cap\Img f=\{0\}$.

Soient :
$$X=\begin{pmatrix}x\\y\\z\end{pmatrix}\in\Img f\cap\Ker f$$
Avec \begin{align*}
        \Ker f&=\vect(0,-1,1) \\
        &=\vect(u)
    \end{align*}

Donc ici, $X\in\Ker f\Leftrightarrow \exists a \in \mathbb{R},\begin{pmatrix}x\\y\\z\end{pmatrix}=a\begin{pmatrix}0\\-1\\1\end{pmatrix}$

De plus, $X\in\Img f\Leftrightarrow \exists \alpha,\beta,\gamma \in \mathbb{R},\\
\begin{pmatrix}x\\y\\z\end{pmatrix}=\begin{pmatrix}\alpha+2\beta+2\gamma\\-\beta-\gamma\\0\end{pmatrix}$

Donc si $X$ vérifie les 2 équations précédentes, on a :
$$
\begin{pmatrix}
    \alpha+2\beta+2\gamma \\
    -\beta-\gamma\\
    0
\end{pmatrix}
=
\begin{pmatrix}0\\-a\\a\end{pmatrix}
$$

Ce qui équivaut au système suivant :
$$
  \left\{\begin{array}{rl}
  \alpha+2\beta+2\gamma & = 0\\
  -\beta-\gamma         & = -a\\
                    0   & = a 
  \end{array}\right.
$$
Comme $a=0$, $X=\begin{pmatrix}0\\0\\0\end{pmatrix}$

Ainsi, on a bien prouvé que $\Img f\cap\Ker f \subset\{0\}$

Cependant, d'après la réciprque du théorème du rang, comme $\Img f$ et $\Ker f$ sont des sous-espaces vectoriels de $\mathbb{R}^3$, ils contiennent l'élément neutre.

On a toujours, $\{0\}\subset\Img f \cap\Ker f$

Donc $\Img f\cap\Ker f=\{0\}$. Ainsi, les 2 sev sont supplémentaires.

$\bullet$ $f$ est-elle une projection ?

Soient les 2 sous-espaces vectoriels suivant :
$$U=\vect\{(0,-1,1)\} \mbox{ et } V=\vect\{(1,0,0);(2,-1,0)\}$$

Calculons $proj^u_v$ :

$$(x,y,z)=a(b_1)+b(b_2)+c(b_3)$$
$$(x,y,z)=a(0,-1,1)+b(1,0,0)+c(2,-1,0)$$


\[
  \left\{\begin{array}{rl}
    x &= 0a+b+2c \\
    y &= -a+0b+-c  \\
    z &= a+0b+0c
  \end{array}\right.
  \Leftrightarrow
  \left\{\begin{array}{rl}
    b &=x-2c \\
    -c &=y+z \\
    a &=z
  \end{array} \right.
\]
\[
  \Leftrightarrow
  \left\{\begin{array}{rl}
      b&=x+2y+2z\\
      c&=-y-z \\
      a&=z
  \end{array}\right.
\]

On obtient : 
$$(x,y,z)=zb_1+(x+2y+2z)b_2+(-y-z)b_3$$
La projection sur $U$ parralèlement à $V$ (où $U=\vect(b_1)$) et $V=\vect\{(b_2,b_3)\}$ vérifie :
\begin{align*}
    proj(b_1)&=b1 \\
    proj(b_2)&=0 \\
    proj(b_3)&=0
\end{align*}

Donc,
\begin{align*}
    proj(x,y,z)&=proj(zb_1+(x+2y+2z)b_2+(-y-z)b_3) \\
    &=zproj(b_1)+(x+2y+2z)proj(b_2)+(-y-z)proj(b_3) \\
    &=z\times b_1
\end{align*}

On peut ainsi exprimer dans la base canonique :  $(0,-z,z)$
$$proj^u_v(x,y,z)=(0,-z,z)$$
Les 3 coordonnées sont colinéaires, c'est une application de rang 1.

$\bullet$ $f$ est-elle une symétrie ?

Pour savoir si $f$ est une symétrie, on calcul $f\circ f$ $(f^2)$.

Calculons donc $A^2$ :

\[A^2=\begin{pmatrix}1&2&2\\0&-1&-1\\0&0&0\end{pmatrix}.\begin{pmatrix}1&2&2\\0&-1&-1\\0&0&0\end{pmatrix}=\begin{pmatrix}1&0&0\\0&1&1\\0&0&0\end{pmatrix}\]
$$f^2\neq Id_3\Rightarrow f \mbox{ n'est pas une symétrie.}$$

%Exercice 4
\section{}

Soit $E$ un $\mathbb{R}$-espace vectoriel et $f$ un endomorphisme de $E$.

$\bullet$ Pour la démonstration, nous allons d'abord supposer que $f\circ p = p\circ f$.

$\forall x \in \Ker(p), p(x) = 0$ par définition du noyau. 

A partir de notre hypothèse :
\begin{align*}
  p \circ f(x) & = f \circ p(x)\\
              & = f(p(x))\\
              & = f(0)\\
              & = 0
\end{align*}
car $f$ est un endomorphisme

Donc $\Ker p$ est stable par $f$.

De même $ \forall y \in \Img p, \exists x \in F, \mbox{ tel que  } y = p(x)$ (définition de l'image de $p$).

D'après notre hypothèse, on peut écrire :

$$f(y) = f(p(x)) = p(f(x))$$

Donc $f(y)$ est antécédent de $f(x)$, donc $f(y) \in \Img p$.

L'image de $p$, $\Img p$ est stable par $f$.

$\bullet$ Réciproque :

Supposons que $\Ker p$ et $\Img p$ sont stables par $f$.

$p$ est une projection donc $F = \Ker p \oplus \Img p$

Soit $x \in \Ker p$ alors $f(p(x)) = f(0) = 0 $ et $p(f(x)) = 0$ car $\Ker p$ stable par $f$.

Soit $y \in \Img p, \exists x \in F, \mbox{ tel que } y = p(x)$.

$$f(p(y)) = f(p(p(x))) = f(p^2(x)) = f(p(x))$$

car $p$ est une projection

donc $f(p(y)) = f(y)$ et $f(y) \in \Img p$ car $\Img p$ est stable par $f$.


\end{document
